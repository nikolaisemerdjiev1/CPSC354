\documentclass{article}

\usepackage{tikz} 
\usetikzlibrary{automata, positioning, arrows} 

\usepackage{amsthm}
\usepackage{amsfonts}
\usepackage{amsmath}
\usepackage{amssymb}
\usepackage{fullpage}
\usepackage{color}
\usepackage{parskip}
\usepackage{hyperref}
  \hypersetup{
    colorlinks = true,
    urlcolor = blue,       % color of external links using \href
    linkcolor= blue,       % color of internal links 
    citecolor= blue,       % color of links to bibliography
    filecolor= blue,        % color of file links
    }
    
\usepackage{listings}
\usepackage[utf8]{inputenc}                                                    
\usepackage[T1]{fontenc}  
\usepackage{quiver}


\definecolor{dkgreen}{rgb}{0,0.6,0}
\definecolor{gray}{rgb}{0.5,0.5,0.5}
\definecolor{mauve}{rgb}{0.58,0,0.82}

\lstset{frame=tb,
  language=haskell,
  aboveskip=3mm,
  belowskip=3mm,
  showstringspaces=false,
  columns=flexible,
  basicstyle={\small\ttfamily},
  numbers=none,
  numberstyle=\tiny\color{gray},
  keywordstyle=\color{blue},
  commentstyle=\color{dkgreen},
  stringstyle=\color{mauve},
  breaklines=true,
  breakatwhitespace=true,
  tabsize=3
}

\newtheoremstyle{theorem}
  {\topsep}   % ABOVESPACE
  {\topsep}   % BELOWSPACE
  {\itshape\/}  % BODYFONT
  {0pt}       % INDENT (empty value is the same as 0pt)
  {\bfseries} % HEADFONT
  {.}         % HEADPUNCT
  {5pt plus 1pt minus 1pt} % HEADSPACE
  {}          % CUSTOM-HEAD-SPEC
\theoremstyle{theorem} 
   \newtheorem{theorem}{Theorem}[section]
   \newtheorem{corollary}[theorem]{Corollary}
   \newtheorem{lemma}[theorem]{Lemma}
   \newtheorem{proposition}[theorem]{Proposition}
\theoremstyle{definition}
   \newtheorem{definition}[theorem]{Definition}
   \newtheorem{example}[theorem]{Example}
\theoremstyle{remark}    
  \newtheorem{remark}[theorem]{Remark}

\title{CPSC-354 Report}
\author{Nikolai Semerdjiev  \\ Chapman University}

\date{\today} 

\begin{document}

\maketitle

\begin{abstract}
\end{abstract}

\setcounter{tocdepth}{3}
\tableofcontents

\section{Introduction}\label{intro}

\section{Week by Week}\label{homework}

\subsection{Week 1}

\subsubsection{Notes and Exploration}

In Week 1, we began with the MIU (or MU) puzzle from Hofstadter’s *Gödel, Escher, Bach*. 
This puzzle introduces the idea of formal systems and rules of inference. 
It is a good starting point to think about what it means to derive a string from an axiom 
under a fixed set of rules.

\begin{center}
    \textbf{The MU Puzzle}
\end{center}

\textbf{RULES:}
\begin{enumerate}
  \item (Append-\texttt{U}) If a string ends with \texttt{I}, you may append \texttt{U}:\\
  \hspace{1em} $x\texttt{I} \to x\texttt{IU}$.
  \item (Double) From \texttt{M}$x$ you may produce \texttt{M}$xx$:\\
  \hspace{1em} $\texttt{M}x \to \texttt{M}xx$.
  \item (III$\to$U) Replace any occurrence of \texttt{III} with \texttt{U}:\\
  \hspace{1em} $x\texttt{III}y \to x\texttt{U}y$.
  \item (Delete \texttt{UU}) Delete any occurrence of \texttt{UU}:\\
  \hspace{1em} $x\texttt{UU}y \to xy$.
\end{enumerate}

\subsubsection{Homework}

\textbf{Problem:} Can you derive \texttt{MU} from \texttt{MI}?

\textbf{Solution:} No, it is impossible to derive \texttt{MU} from \texttt{MI}. At first, 
playing around with the rules, I noticed that the goal was to create the correct number 
of \texttt{I}'s so that they could be converted to a single \texttt{U}. This means we 
would need $N_I \bmod 3 = 0$, where $N_I$ counts the \texttt{I}'s.

Now, consider how each rule affects $N_I$:
\begin{itemize}
  \item \textbf{(Append-\texttt{U})} Appends a \texttt{U}, leaves $N_I$ unchanged.
  \item \textbf{(Double)} $\texttt{M}x \to \texttt{M}xx$ doubles $N_I$; in modular arithmetic, $N_I \mapsto 2N_I \pmod{3}$.
  \item \textbf{(III$\to$U)} Removes three \texttt{I}'s, leaving $N_I \bmod 3$ unchanged.
  \item \textbf{(Delete \texttt{UU})} Only touches \texttt{U}'s, leaves $N_I$ unchanged.
\end{itemize}

Starting from $\texttt{MI}$, we have $N_I = 1 \equiv 1 \pmod{3}$. Doubling cycles 
between $1$ and $2$ modulo $3$, never producing $0$. Thus, it is impossible to reach 
$N_I \equiv 0 \pmod{3}$.

Since $\texttt{MU}$ has $N_I=0$, it cannot be derived from $\texttt{MI}$.

\subsubsection{Questions}

What is the reasoning behind being able to convert $\texttt{MIII}$ into $\texttt{MU}$ 
(using the rule $\texttt{III}\to \texttt{U}$) but not being able to go the other way 
(from $\texttt{MU}$ to $\texttt{MIII}$)?

\subsection{Week 2}

\subsubsection{Homework}

\textbf{Problem:} Consider the following ARSs. Draw a picture for each one. 
Are the ARSs terminating? Are they confluent? Do they have unique normal forms?

\begin{enumerate}
  \item $A = \{\}, \quad R = \{\}$
  \item $A = \{a\}, \quad R = \{\}$
  \item $A = \{a\}, \quad R = \{(a,a)\}$
  \item $A = \{a,b,c\}, \quad R = \{(a,b),(a,c)\}$
  \item $A = \{a,b\}, \quad R = \{(a,a),(a,b)\}$
  \item $A = \{a,b,c\}, \quad R = \{(a,b),(b,b),(a,c)\}$
  \item $A = \{a,b,c\}, \quad R = \{(a,b),(b,b),(a,c),(c,c)\}$
\end{enumerate}

\textbf{Solution:}  
For each ARS, I drew a graph (see figures) and analyzed:

- **Termination:** whether there are infinite chains.  
- **Confluence:** whether every divergence can rejoin.  
- **Unique normal forms:** whether each element has a unique NF.

I will include one diagram for each ARS along with a short explanation of my analysis.

\paragraph{ARS 1: $A = \{\}, \; R = \{\}$}

There is no infinite chain, no diverging paths exist as there is no path at all, and 
nothing exists to violate uniqueness. Therefore $\checkmark$ terminating, $\checkmark$ confluent, 
and $\checkmark$ has unique normal form.

\paragraph{ARS 2: $A = \{a\}, \; R = \{\}$}

% https://q.uiver.app/#q=WzAsMSxbMCwwLCJcXGJ1bGxldCJdXQ==
\[\begin{tikzcd}
	\bullet
\end{tikzcd}\]

No rewrite steps: no infinite chain, no diverging paths exist or no path at all so it is vacuously satisfied and $a$ is the only reachable normal form from $a$. Therefore $\checkmark$ terminating, $\checkmark$ confluent, and $\checkmark$ has unique normal form.

\paragraph{ARS 3: $A=\{a\},\; R=\{(a,a)\}$}

% https://q.uiver.app/#q=WzAsMSxbMCwwLCJcXGJ1bGxldCJdLFswLDBdXQ==
\[\begin{tikzcd}
	\bullet
	\arrow[from=1-1, to=1-1, loop, in=55, out=125, distance=10mm]
\end{tikzcd}\]

There is an infinite chain, $a \to^* a$. Since $a \to^* y$ and $a \to^* z$, therefore $y = z = a$, joining at $a$, and there are no normal forms as there is only one term, $a$, which has infinite outgoing rewrite steps. Therefore $\times$ terminating, $\checkmark$ confluent, and $\times$ unique normal form.

\paragraph{ARS 4: $A=\{a,b,c\},\; R=\{(a,b),(a,c)\}$}

% https://q.uiver.app/#q=WzAsMyxbMSwwLCJcXGJ1bGxldCJdLFswLDEsIlxcYnVsbGV0Il0sWzIsMSwiXFxidWxsZXQiXSxbMCwxXSxbMCwyXV0=
\[
\begin{tikzcd}
	& \bullet \\
	\bullet && \bullet
	\arrow[from=1-2, to=2-1]
	\arrow[from=1-2, to=2-3]
\end{tikzcd}
\]

There is no infinite chain, two diverging paths that do not get joined, and two different normal forms from $a$. Therefore $\checkmark$ terminating, $\times$ confluent, and $\times$ unique normal form.

\paragraph{ARS 5: $A=\{a,b\},\; R=\{(a,a),(a,b)\}$}

% https://q.uiver.app/#q=WzAsMixbMCwwLCJcXGJ1bGxldCJdLFsxLDAsIlxcYnVsbGV0Il0sWzAsMF0sWzAsMV1d
\[
\begin{tikzcd}
	\bullet & \bullet
	\arrow[from=1-1, to=1-1, loop, in=55, out=125, distance=10mm]
	\arrow[from=1-1, to=1-2]
\end{tikzcd}
\]

Even with one infinite chain it does not terminate, $a\to a$ and $a\to b$ and they join at $b$, $b$ is the only normal form since $a$ has an infinite chain therefore $\times$ terminating, $\checkmark$ confluent, has a $\checkmark$ unique normal form.


\paragraph{ARS 6: $A=\{a,b,c\},\; R=\{(a,b),(b,b),(a,c)\}$}

% https://q.uiver.app/#q=WzAsMyxbMSwwLCJcXGJ1bGxldCJdLFswLDEsIlxcYnVsbGV0Il0sWzIsMSwiXFxidWxsZXQiXSxbMCwxXSxbMSwxXSxbMCwyXV0=
\[
\begin{tikzcd}
	& \bullet \\
	\bullet && \bullet
	\arrow[from=1-2, to=2-1]
	\arrow[from=1-2, to=2-3]
	\arrow[from=2-1, to=2-1, loop, in=55, out=125, distance=10mm]
\end{tikzcd}
\]

Infinite chain at $b$ (since $b \to b \to b \cdots$). From $a$ the system diverges to $b$ and $c$ with no connecting path to join them. The only normal form is $c$, but not every element reduces to a unique normal form, so the system is $\times$ terminating, $\times$ confluent, and $\times$ has unique normal form.

\paragraph{ARS 7: $A=\{a,b,c\},\; R=\{(a,b),(a,c),(b,b),(c,c)\}$}

% https://q.uiver.app/#q=WzAsMyxbMSwwLCJcXGJ1bGxldCJdLFswLDEsIlxcYnVsbGV0Il0sWzIsMSwiXFxidWxsZXQiXSxbMCwxXSxbMCwyXSxbMSwxXSxbMiwyXV0=
\[
\begin{tikzcd}
	& \bullet \\
	\bullet && \bullet
	\arrow[from=1-2, to=2-1]
	\arrow[from=1-2, to=2-3]
	\arrow[from=2-1, to=2-1, loop, in=55, out=125, distance=10mm]
	\arrow[from=2-3, to=2-3, loop, in=55, out=125, distance=10mm]
\end{tikzcd}
\]

There are two infinite chains ($b \to b \to \cdots$ and $c \to c \to \cdots$). 
From $a$ the system diverges to $b$ and $c$, but since $b$ only reaches $b$ and $c$ only reaches $c$, 
the peak at $a$ does not join. Every element has outgoing rewrite steps, so there are no normal forms. 
Therefore $\times$ terminating, $\times$ confluent, and $\times$ has unique normal form.

\textbf{All Eight Combinations}

The homework also asked to find examples of ARSs for each of the eight possible
combinations of confluence, termination, and unique normal forms. 
The table below summarizes the results.

\begin{center}
\begin{tabular}{|c|c|c|c|}
\hline
\textbf{Confluent} & \textbf{Terminating} & \textbf{Unique Normal Forms} & \textbf{Example $(A,R)$} \\
\hline
True  & True  & True  & $A = \{a,b,c,d\},\; R = \{(a,b),(a,c),(b,d),(c,d)\}$ \\
\hline
True  & True  & False & N/A \\
\hline
True  & False & True  & $A = \{a,b,c,d\},\; R = \{(a,a),(a,b),(a,c),(b,d),(c,d)\}$ \\
\hline
True  & False & False & $A = \{a,b,c,d\},\; R = \{(a,b),(a,c),(b,d),(c,d),(d,d)\}$ \\
\hline
False & True  & True  & N/A \\
\hline
False & True  & False & $A = \{a,b,c,d\},\; R = \{(a,b),(a,c),(b,d)\}$ \\
\hline
False & False & True  & $A = \{a,b,c\},\; R = \{(a,b),(a,c),(b,d),(c,d),(d,d)\}$ \\
\hline
False & False & False & $A = \{a,b,c\},\; R = \{(a,a),(a,b),(a,c)\}$ \\
\hline
\end{tabular}
\end{center}

\textbf{Example Graphs}

\paragraph{Graph 1}
% https://q.uiver.app/#q=WzAsNCxbMSwwLCJcXGJ1bGxldCJdLFswLDEsIlxcYnVsbGV0Il0sWzEsMiwiXFxidWxsZXQiXSxbMiwxLCJcXGJ1bGxldCJdLFswLDFdLFswLDNdLFsxLDJdLFszLDJdXQ==
\[\begin{tikzcd}
	& \bullet \\
	\bullet && \bullet \\
	& \bullet
	\arrow[from=1-2, to=2-1]
	\arrow[from=1-2, to=2-3]
	\arrow[from=2-1, to=3-2]
	\arrow[from=2-3, to=3-2]
\end{tikzcd}\]

\paragraph{Graph 2}
This comination is impossible as termination gives a normal form for each element AND confluence guarantees any two reduction paths from the same element to join therefore all reductions end at some normal form.

\paragraph{Graph 3}
% https://q.uiver.app/#q=WzAsNCxbMSwwLCJcXGJ1bGxldCJdLFswLDEsIlxcYnVsbGV0Il0sWzEsMiwiXFxidWxsZXQiXSxbMiwxLCJcXGJ1bGxldCJdLFswLDBdLFswLDFdLFsxLDJdLFswLDNdLFszLDJdXQ==
\[\begin{tikzcd}
	& \bullet \\
	\bullet && \bullet \\
	& \bullet
	\arrow[from=1-2, to=1-2, loop, in=55, out=125, distance=10mm]
	\arrow[from=1-2, to=2-1]
	\arrow[from=1-2, to=2-3]
	\arrow[from=2-1, to=3-2]
	\arrow[from=2-3, to=3-2]
\end{tikzcd}\]

\paragraph{Graph 4}
% https://q.uiver.app/#q=WzAsNCxbMSwwLCJcXGJ1bGxldCJdLFswLDEsIlxcYnVsbGV0Il0sWzEsMiwiXFxidWxsZXQiXSxbMiwxLCJcXGJ1bGxldCJdLFswLDNdLFswLDFdLFsxLDJdLFszLDJdLFsyLDJdXQ==
\[\begin{tikzcd}
	& \bullet \\
	\bullet && \bullet \\
	& \bullet
	\arrow[from=1-2, to=2-1]
	\arrow[from=1-2, to=2-3]
	\arrow[from=2-1, to=3-2]
	\arrow[from=2-3, to=3-2]
	\arrow[from=3-2, to=3-2, loop, in=55, out=125, distance=10mm]
\end{tikzcd}\]

\paragraph{Graph 5}

This case is impossible. Every system that is terminating and where every element has a unique normal form must also be confluent. 
Indeed, if $a \to^* y$ and $a \to^* z$, then both $y$ and $z$ reduce to some normal forms. 
Since the normal form is unique, $y$ and $z$ must reduce to the same normal form, 
which means the system is confluent.


\paragraph{Graph 6}
% https://q.uiver.app/#q=WzAsNCxbMSwwLCJcXGJ1bGxldCJdLFswLDEsIlxcYnVsbGV0Il0sWzAsMiwiXFxidWxsZXQiXSxbMiwxLCJcXGJ1bGxldCJdLFswLDFdLFsxLDJdLFswLDNdXQ==
\[\begin{tikzcd}
	& \bullet \\
	\bullet && \bullet \\
	\bullet
	\arrow[from=1-2, to=2-1]
	\arrow[from=1-2, to=2-3]
	\arrow[from=2-1, to=3-1]
\end{tikzcd}\]

\paragraph{Graph 7}
% https://q.uiver.app/#q=WzAsMyxbMSwwLCJcXGJ1bGxldCJdLFswLDEsIlxcYnVsbGV0Il0sWzIsMSwiXFxidWxsZXQiXSxbMCwwXSxbMCwxXSxbMCwyXSxbMiwyXV0=
\[\begin{tikzcd}
	& \bullet \\
	\bullet && \bullet
	\arrow[from=1-2, to=1-2, loop, in=55, out=125, distance=10mm]
	\arrow[from=1-2, to=2-1]
	\arrow[from=1-2, to=2-3]
	\arrow[from=2-3, to=2-3, loop, in=55, out=125, distance=10mm]
\end{tikzcd}\]

\paragraph{Graph 8}
% https://q.uiver.app/#q=WzAsMyxbMSwwLCJcXGJ1bGxldCJdLFsyLDEsIlxcYnVsbGV0Il0sWzAsMSwiXFxidWxsZXQiXSxbMCwwXSxbMCwyXSxbMCwxXV0=
\[\begin{tikzcd}
	& \bullet \\
	\bullet && \bullet
	\arrow[from=1-2, to=1-2, loop, in=55, out=125, distance=10mm]
	\arrow[from=1-2, to=2-1]
	\arrow[from=1-2, to=2-3]
\end{tikzcd}\]



\subsubsection{Questions}

How can thinking about ARSs help us better understand the way programming languages define and control the process of evaluating programs?


\section{Essay}

\section{Evidence of Participation}

\section{Conclusion}\label{conclusion}

\begin{thebibliography}{99}
\bibitem[BLA]{bla} Author, \href{https://en.wikipedia.org/wiki/LaTeX}{Title}, Publisher, Year.
\end{thebibliography}

\end{document}