\documentclass{article}

\usepackage{tikz} 
\usetikzlibrary{automata, positioning, arrows} 

\usepackage{amsthm}
\usepackage{amsfonts}
\usepackage{amsmath}
\usepackage{amssymb}
\usepackage{fullpage}
\usepackage{color}
\usepackage{parskip}
\usepackage{hyperref}
  \hypersetup{
    colorlinks = true,
    urlcolor = blue,       % color of external links using \href
    linkcolor= blue,       % color of internal links 
    citecolor= blue,       % color of links to bibliography
    filecolor= blue,        % color of file links
    }
    
\usepackage{listings}
\usepackage[utf8]{inputenc}                                                    
\usepackage[T1]{fontenc}                                                       

\definecolor{dkgreen}{rgb}{0,0.6,0}
\definecolor{gray}{rgb}{0.5,0.5,0.5}
\definecolor{mauve}{rgb}{0.58,0,0.82}

\lstset{frame=tb,
  language=haskell,
  aboveskip=3mm,
  belowskip=3mm,
  showstringspaces=false,
  columns=flexible,
  basicstyle={\small\ttfamily},
  numbers=none,
  numberstyle=\tiny\color{gray},
  keywordstyle=\color{blue},
  commentstyle=\color{dkgreen},
  stringstyle=\color{mauve},
  breaklines=true,
  breakatwhitespace=true,
  tabsize=3
}

\newtheoremstyle{theorem}
  {\topsep}   % ABOVESPACE
  {\topsep}   % BELOWSPACE
  {\itshape\/}  % BODYFONT
  {0pt}       % INDENT (empty value is the same as 0pt)
  {\bfseries} % HEADFONT
  {.}         % HEADPUNCT
  {5pt plus 1pt minus 1pt} % HEADSPACE
  {}          % CUSTOM-HEAD-SPEC
\theoremstyle{theorem} 
   \newtheorem{theorem}{Theorem}[section]
   \newtheorem{corollary}[theorem]{Corollary}
   \newtheorem{lemma}[theorem]{Lemma}
   \newtheorem{proposition}[theorem]{Proposition}
\theoremstyle{definition}
   \newtheorem{definition}[theorem]{Definition}
   \newtheorem{example}[theorem]{Example}
\theoremstyle{remark}    
  \newtheorem{remark}[theorem]{Remark}

\title{CPSC-354 Report}
\author{Nikolai Semerdjiev  \\ Chapman University}

\date{\today} 

\begin{document}

\maketitle

\begin{abstract}
\end{abstract}

\setcounter{tocdepth}{3}
\tableofcontents

\section{Introduction}\label{intro}

\section{Week by Week}\label{homework}

\subsection{Week 1}

\begin{center}
    \textbf{The MU Puzzle}
\end{center}

\textbf{RULES:}

\begin{enumerate}
  \item (Append-\texttt{U}) If a string ends with \texttt{I}, you may append \texttt{U}:\
  \hspace{1em} $x\texttt{I} \to x\texttt{IU}$.
  \item (Double) From \texttt{M}$x$ you may produce \texttt{M}$xx$:\
  \hspace{1em} $\texttt{M}x \to \texttt{M}xx$.
  \item (III$\to$U) Replace any occurrence of \texttt{III} with \texttt{U}:\
  \hspace{1em} $x,\texttt{III},y \to x,\texttt{U},y$.
  \item (Delete \texttt{UU}) Delete any occurrence of \texttt{UU}:\
  \hspace{1em} $x,\texttt{UU},y \to x,y$.
\end{enumerate}

\textbf{Question:} Can you go from $\texttt{MI} \to \texttt{MU}$?

\textbf{Solution:} No, it is impossible to derive \texttt{MU} from \texttt{MI}. At first, playing around with the rules, I noticed that the goal was to create the correct number of \texttt{I}'s where they would be converted to a single \texttt{U}, which means that we needed the rules to create $N_I \bmod 3 = 0$ as $N_I$ is \texttt{I} count.

Now, consider how each rule affects $N_I$:
\begin{itemize}
  \item \textbf{(Append-\texttt{U})} This rule appends a \texttt{U} to the end of the 
  string. It does not change the number of \texttt{I}'s, so $N_I$ is unchanged.
  
  \item \textbf{(Double)} The rule $\texttt{M}x \to \texttt{M}xx$ doubles the part 
  after \texttt{M}. If the first string has $N_I$ \texttt{I}'s, then the new one has 
  $2N_I$ \texttt{I}'s. In modular arithmetic, $N_I$ is multiplied by $2$ modulo $3$.
  
  \item \textbf{(III$\to$U)} This rule removes exactly three \texttt{I}'s. Therefore 
  $N_I \mapsto N_I - 3$, which leaves $N_I \bmod 3$ unchanged.
  
  \item \textbf{(Delete \texttt{UU})} This rule affects only \texttt{U}'s, not 
  \texttt{I}'s, so $N_I$ is unchanged.
\end{itemize}

Starting from $\texttt{MI}$, we have $N_I=1$, so $N_I \equiv 1 \pmod{3}$. 
The only nontrivial change is doubling, which cycles $1 \mapsto 2 \mapsto 1 \mapsto 2 \dots$ 
but never produces $0$. Thus, it is impossible to reach $N_I \equiv 0 \pmod{3}$.

Since $\texttt{MU}$ has $N_I=0$, it is not derivable from $\texttt{MI}$.

\section{Essay}

\section{Evidence of Participation}

\section{Conclusion}\label{conclusion}

\begin{thebibliography}{99}
\bibitem[BLA]{bla} Author, \href{https://en.wikipedia.org/wiki/LaTeX}{Title}, Publisher, Year.
\end{thebibliography}

\end{document}